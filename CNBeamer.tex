\documentclass{beamer}
\mode<presentation>{\usetheme{boxes}}
\usepackage{graphicx} % Required for including pictures
\usepackage{wrapfig} % Allows in-line images
\usepackage{amsmath}
\usepackage{enumerate} 
\usepackage{mathpazo} % Use the Palatino font
\usepackage[framed,numbered]{matlab-prettifier} %include code sections
\usepackage[T1]{fontenc} % Required for accented characters
\usepackage[utf8x]{inputenc} % Enecoding used in the document
\usepackage[spanish]{babel}
\title{Solución numérica de ecuaciones diferenciales parciales parabólicas por el método de Crank-Nicolson.}
\author{Duban Arley Medina Eusse, Roberto L. Taborda}

\begin{document}
\begin{frame}
	\titlepage
\end{frame}

\begin{frame}{Introducción}
	Nuestro problema:
	$$\frac{\partial u}{\partial t}=\alpha^2\frac{\partial^2 u}{\partial x^2}$$
\end{frame}
\begin{frame}{Introducción}
Anteriormente conociamos dos metodos para hallar una solucion numerica, las soluciones de diferencias finitas hacia adelante $$\frac{u_{n,m+1}-u_{n,m}}{k}-\alpha^2\frac{u_{n+1,m}-2u_{n,m}+u_{n-1,m}}{h^2}=0$$, y hacia atras $$\frac{u_{n,m}-u_{n,m-1}}{k}-\alpha^2\frac{u_{n+1,m}-2u_{n,m}+u_{n-1,m}}{h^2}=0$$.
\end{frame}
\begin{frame}{Métodos Previos}
Las anteriores soluciones numericas poseen ciertos inconvenientes para la solucion de nuestro problema.
\begin{itemize}
\item El metodo de diferencias hacia adelante:
\begin{itemize}
 	\item Posee problemas de estabilidad/convergencia cuando se enfrenta a valores de $\lambda > \frac{1}{2}$.
\end{itemize}
\item El metodo de diferencias hacia atras:
\begin{itemize}
\item Surge como solucion para el problema de estabilidad.
\item Posee error temporal de grado 1.
\end{itemize}
\end{itemize}
\end{frame}
\begin{frame}{Crank-Nicolson}
EL metodo crank nicolson surge entonces para poder dar solucion a dichos problemas:
\begin{itemize}
	\item El error en el tiempo para este metodo es $(\varDelta t)^2$.
	\item No posee restricciones de estabilidad.
\end{itemize}
Por otro lado este método es implicito, es decir que se hace necesarion calcular los valores completos de $(t+\varDelta t)$ para calcular los valores en $(t)$.
\end{frame}
\begin{frame}{Desarrollo matemático}
El probrema a solucionar esta dado por la ecuación diferencial parcial:
$$\frac{\partial u}{\partial t}=\alpha^2\frac{\partial^2 u}{\partial x^2}$$
Bajo las siguientes condiciones iniciales y de frontera:
$$u(x,0)=f(x)$$
$$u(0,t)=0\hspace{1cm}u(l,t)=0$$
\end{frame}
\begin{frame}{Desarollo matemático}
Escribiendo la ecuación de las diferencias hacia atras para m + 1 y realizando la media aritmética entre el método de las diferencias hacia atrás y hacia adelante se obtiene: 
$$\frac{u_{n,m+1}-u_{n,m}}{k}-$$
$$\frac{\alpha^2}{2}(\frac{u_{n+1,m}-2u_{n,m}+u_{n-1,m}}{h^2}+\frac{u_{n+1,m+1}-2u_{n,m+1}+u_{n-1,m+1}}{h^2})=0$$
La anterior ecuación es la correspondiente al método de Crank-nicolson.
\end{frame}
\begin{frame}{Forma matricial del método}
Para llevar la anterior ecuación a su forma matricial es necesario definir:
$$\lambda=\dfrac{\alpha^{2}k}{h^{2}}$$
Al aplicarlo a la ecuacion del método:
$$(1+\lambda)u_{n,m+1}-\frac{\lambda}{2}u_{n+1,m+1}-\frac{\lambda}{2}u_{n-1,m+1}=$$
$$(1-\lambda)u_{n,m}+\frac{\lambda}{2}u_{n+1,m}+\frac{\lambda}{2}u_{n-1,m}$$
y
$$u^{(m)}=(u_{1,m},u_{2,m},\ldots,u_{N-1,m})$$
$$Au^{(m+1)}=Bu^{(m)}$$
\end{frame}
\begin{frame}{Forma matricial del método}
Donde $A$ y $B$ son matrices tridiagonales:
$$
A=
\begin{bmatrix}
  1+\lambda  & -\frac{\lambda}{2} & 0 & \ldots\\ 
 -\frac{\lambda}{2} & 1+\lambda & -\frac{\lambda}{2} & \ldots\\ 
 0 & -\frac{\lambda}{2} & 1+\lambda & \ldots\\ 
 \vdots & \vdots & \vdots & \ddots 
\end{bmatrix}
$$
$$
B=
\begin{bmatrix}
 1-\lambda  & \frac{\lambda}{2} & 0 & \ldots\\ 
 \frac{\lambda}{2} & 1-\lambda & \frac{\lambda}{2} & \ldots\\ 
 0 & \frac{\lambda}{2} & 1-\lambda & \ldots\\ 
 \vdots & \vdots & \vdots & \ddots 
\end{bmatrix}
$$
\end{frame}
\begin{frame}{Implementación Computacional}
Inicialmene definamos los argumentos de nuestra función:
\begin{itemize}
	\item $L$ (Escalar): Limite superior de la variable espacial $(x)$.
	\item $T$ (Escalar): Limite superior de la variable temporal $(t)$.
	\item $alpha$ (Escalar): Raiz cuadradea de el coeficiente que acompaña a $\frac{\partial^2 u}{\partial x^2}$
	\item $n$ (Entero): Número de intervalos temporales a usar.
	\item $m$ (Entero): Número de intervalos espaciales a usar.
	\item $funk$ (Function handle): Función que representa el estado inicial de $u(x,t)=u(x,0)$ 
\end{itemize}
\end{frame}
\begin{frame}{Implementación Computacional}
\framesubtitle{Definición}
Primero definimos la función:
\begin{itemize}
	\item \lstinline[style=Matlab-editor]!function w = crnkNclsn(L,T,alpha,m,n,funk)!
\end{itemize} 
Despues debemos definir $h$, $k$, y $lamdba$
\begin{itemize}
	\item \lstinline[style=Matlab-editor]!h = L./m;!
	\item \lstinline[style=Matlab-editor]!k = T./n;!
	\item \lstinline[style=Matlab-editor]!lambda = (alpha.^2).*k./(h.^2);!
\end{itemize}
\end{frame}
\begin{frame}{Implementación Computacional}
\framesubtitle{Construcción de matrices}
Inicializamos los valores de la primera columna($f(x)=u(x,0)$) de la matriz que representa nuestra aproximación:
\begin{itemize}
	\item \lstinline[style=Matlab-editor]!w(2:m,1) = funk(h.*(1:(m-1)).')!
\end{itemize}
Y construmios las matrices $A$ y $B$ definidas anteriormente:\newline
Sabemos que las matrices son tridiagonales asi que primero construiremos sus diagonales como las columnas de otras matrices:
\begin{itemize}
	\item \lstinline[style=Matlab-editor]!Adiag = repmat([-lambda./2 1+lambda -lambda./2],m-1,1);!
	\item \lstinline[style=Matlab-editor]!Bdiag = repmat([lambda./2 1-lambda lambda./2],m-1,1);!
\end{itemize}
\end{frame}

\begin{frame}{Implementación Computacional}
\framesubtitle{Construcción de matrices}
Pero de que nos sirve representar las diagonales que necesitamos a manera de columnas?\newline
Facilmente podemos convertir dichas columnas en las diagonales que queramos de una matriz sparse con el siguente comando:
\begin{itemize}
	\item \lstinline[style=Matlab-editor]!A = spdiags(Adiag,[-1 0 1],m-1,m-1)!
	\item \lstinline[style=Matlab-editor]!B = spdiags(Bdiag,[-1 0 1],m-1,m-1)!
\end{itemize}
\end{frame}
\begin{frame}{Implementación Computacional}
\framesubtitle{Solución del sistema de ecuaciones.}
Una vez se tienen definidas las matrices el problema se resume en 
$$Au^{(m+1)}=Bu^{(m)}$$
Para ello un solo ciclo debe ser suficiente y gracias a la facilidad que posee el software para manipular matrices la solución del sistema se reduce a:\newline
\lstinline[style=Matlab-editor]!for i = 1: (n-1)!\newline
\lstinline[style=Matlab-editor]!\	w(2:m,i+1) = A\\(B*w(2:m,i));!\newline
\lstinline[style=Matlab-editor]!end!
\end{frame}
\begin{frame}{Implementación Computacional}
\framesubtitle{Presentación de los resultados}
Finalmente para presentar los resultados de nuestra aproximacion volvemos nuestra matriz una matriz completa:
\begin{itemize}
	\item \lstinline[style=Matlab-editor]!w = full(w);!
\end{itemize}
Adicionalmente si se requiere se puede presentar la evolucion temporal de nuestra aproximacion:
\begin{itemize}
	\item \lstinline[style=Matlab-editor]!x = linspace(0,L,m+1);! \newline
	\lstinline[style=Matlab-editor]!for i=1:n! \newline
	\lstinline[style=Matlab-editor]!plot(x,w(:,i)')! \newline
	\lstinline[style=Matlab-editor]!F(i)=getframe(1)! \newline
	\lstinline[style=Matlab-editor]!end! \newline
\end{itemize}
\end{frame}
\begin{frame}
\end{frame}
\begin{frame}
\end{frame}
\begin{frame}
\end{frame}
\end{document}